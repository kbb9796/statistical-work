\documentclass{article}
\usepackage{amsmath,amsthm,amssymb, mathtools}
\usepackage{geometry} % Page layout
\geometry{a4paper, margin=1in} % Adjust margins if needed

\begin{document}
Before we get into the proof, we need to remember two things. \\
\textbf{Remember:} a measure preserving transformation is some function $T : X \rightarrow X$ on a set X such that
$\mu(x) = \mu(T^{-1}x)$ for $x \in X$.
\\
\textbf{Also remember:} an element $x \in B$ is said to be \textit{B-recurrent} if there exists some $k \geq 1$
such that $T^{k}x \in B$. 
\\
\textbf{Claim.} For probability space ($X$, $\sigma$, $\mu$), if $B \in \sigma$ and $\mu(B) > 0$, then almost every $x \in B$ is B-recurrent, i.e. returns to B at least once. 
\begin{proof}
Let $F$ be the set of all elements in $B$ that are not B-current.
We want to show that $\mu(F) = 0$. We have $$ F = \{x \in B : T^{k}x \not\in B\ \mathrm{for\ any\ k \geq 1\}}.$$ 
So $F \cap T^{-k}x = \emptyset$ for any $k \geq 1$. By construction, you can't reach a point in $F$ from another point in $F$ via transformations. 
It follows that $T^{-l}F \cap T^{-m}F = \emptyset$ for $l, m \geq 1$ and $l \neq m$. Concretely, the preimages of $F$ under at least one transformation $T$ 
don't share any elements (if the preimages did share elements, it would mean that you could get from $F$ back into $F$ in $|m - l|$ transformations, but
we've constructed $F$ precisely so you can't return to it). 
Thus, $F$, $T^{-1}F$, $T^{-2}F$, $...$ are pairwise disjoint. Since T is measure preserving, we have $\mu(F) = \mu(T^{-k})$ for any $k \geq 1$. 
Now we assume that $\mu(F) > 0$ to produce a contradiction, ultimatley showing that $\mu(F) = 0$. 
If $\mu(F) > 0$, then $$\mu(\bigcup\limits_{k=1}^{\infty} T^{-k}F) \leq \sum\limits_{k = 1}^{\infty} \mu(T^{-k}F) = \infty$$ by countable subadditivity.
Since we're working on a probability space, we know $$\bigcup\limits_{k=1}^{\infty}T^{-k}F \subseteq X,$$ and by taking measures we get the contradiction
$$1 = \mu(X) \geq \mu(\bigcup\limits_{k=1}^{\infty}T^{-k}F) = \infty.$$ By contradiction, it must be that $\mu(F) = 0$. 
Almost every element of $B$ is B-recurrent. 

\end{proof}
\end{document}