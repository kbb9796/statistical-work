\documentclass{article}
\usepackage{amsmath,amsthm,amssymb}
\usepackage{geometry} % Page layout
\geometry{a4paper, margin=1in} % Adjust margins if needed

\begin{document}
The Poincare Recurrence Theorem implies that almost every element in $B$ returns to $B$ infinitely often. \\ \\
\textbf{Claim.} There exists infinitely many integers $n_1 < n_2 < n_3 ...$ such that for $x \in B$ we have
$T^{n_i}x \in B$ almost everywhere. 
\begin{proof}
Let $$D = \{x \in B : T^{k}x \in B\ \mathrm{for\ finitely\ many\ k \geq 1\}}.$$ 
Meaning, to be in $D$ you are an element of the set $B$ such that after enough transformations you never return to $B$. 
We can rewrite the condition of $D$ in terms of our familiar set $F$ comprised of the elements of $B$ that never return to $B$, 
$$D = \{x \in B\ : T^{k}x \in F\ \mathrm{for\ some\ k \geq 0\}}.$$
In words, if you are an element of $D$, you are an element of $B$ with the extra condition that you never return to $B$ after some finite amount
of transformations. Therefore $$D \subseteq \bigcup\limits_{k=0}^{\infty} T^{-k}F.$$ Thus, by monotonicity of measure, we have
$$ \mu(D) \leq \mu(\bigcup\limits_{k=0}^{\infty} T^{-k}F) = 0$$ (where we proved the union of the preimages of repeated transformations had measure 0 
in the proof of Poincare's Recurrence Theorem).
Hence, almost all $x \in B$ return to 
$B$ infinitely often. 

\end{proof}
\end{document}