\documentclass{article}
\usepackage{amsmath,amsthm,amssymb}
\usepackage{geometry} % Page layout
\geometry{a4paper, margin=1in} % Adjust margins if needed

\begin{document}

The Borel-Cantelli lemma is one of my favorite results in measure and probability theory. And the proof is particularly elegant to me. 

\textbf{Claim.} For finite measure space $(X, \Sigma, \mu)$, if $\{E_n\}_{n=1}^{\infty}$ $\subset$ $\Sigma$ and 
$\sum\limits_{n=1}^{\infty} \mu(E_n) < \infty$, \\ then $\mu(\limsup\limits_{n\rightarrow\infty} E_n) = 0$.

\begin{proof}
By definition, we have $$\limsup\limits_{n\rightarrow\infty} E_n = \bigcap\limits_{n=1}^{\infty} \bigcup\limits_{k=n}^{\infty} E_k.$$
To tidy things up, we define $$ A_n := \bigcup\limits_{k = n}^{\infty} E_k,$$ so, substituting we get 
$$\limsup\limits_{n\rightarrow\infty} E_n = \bigcap\limits_{n=1}^{\infty} A_n.$$ 
Of course, since we are only removing elements from $\{A_n\}$ as $n$ increases, we have
$A_1 \supset A_2 \supset A_3 ... $. Thus, by continuity from above--a basic property of measure--we can write 

\begin{align*}
    \mu(\limsup\limits_{n\rightarrow\infty} E_n) &= \mu(\bigcap\limits_{n=1}^{\infty} A_n) \\
    &= \lim_{n\rightarrow\infty} \mu(A_n) &&\text{(by continuity from above)}\\
    &= \lim_{n\rightarrow\infty} \mu(\bigcup\limits_{k = n}^{\infty} E_k) &&\text{(substituting back)} \\ 
    &\leq \lim_{n\rightarrow\infty}(\sum\limits_{k = n}^{\infty} \mu(E_k)) &&\text{(by countable subadditivity property of measure)} \\
    &= 0.
\end{align*}
Note that the last step follows from the fact that $\sum\limits_{n=1}^{\infty} \mu(E_n)$ converges. Meaning, for any $\epsilon > 0$, 
there exists some $N$ such that for each $k \geq N$, we have $\sum\limits_{k = N}^{\infty} \mu(E_k) < \epsilon$. 
Hence, we can make the tail end of the sum arbitrarily small. Therefore, the limit in the second to last line converges to 0. 
And lastly, by non-negativity of measure, we have the limit superior bounded above and below by 0. Thus, \\
$\mu(\limsup\limits_{n\rightarrow\infty} E_n) = 0$. $\qedhere$
\end{proof}
\end{document}