\documentclass{article}
\usepackage{amsmath,amsthm,amssymb}
\usepackage{geometry} % Page layout
\geometry{a4paper, margin=1in} % Adjust margins if needed

\begin{document}

\textbf{Claim.} For finite measure space $(X, \Sigma, \mu)$, if $\{E_n\}_{n=1}^{\infty}$ $\subset$ $\Sigma$ and 
$\sum\limits_{n=1}^{\infty} \mu(E_n) < \infty$, \\ then $\mu(\limsup\limits_{n\rightarrow\infty} E_n) = 0$.

\begin{proof}
By definition, we have $$\limsup\limits_{n\rightarrow\infty} E_n = \bigcap\limits_{n=1}^{\infty} \bigcup\limits_{k=n}^{\infty} E_k.$$
To tidy things up, we define the inner union as $$ A_n := \bigcup\limits_{k = n}^{\infty} E_k$$.
Substituting our newly defined $A_n$ into the equation, we can write. 
$$\limsup\limits_{n\rightarrow\infty} E_n = \bigcap\limits_{n=1}^{\infty} A_n.$$ 
Now we note the descending structure of $A_n$. Since we only remove elements from $\{A_n\}$ as $n$ increases, we have
$A_1 \supset A_2 \supset A_3 ... $. Thus, by continuity from above (a property of measure) we can write 

\begin{align*}
    \mu(\limsup\limits_{n\rightarrow\infty} E_n) &= \mu(\bigcap\limits_{n=1}^{\infty} A_n) \\
    &= \lim_{n\rightarrow\infty} \mu(A_n) &&\text{(by continuity from above)}\\
    &= \lim_{n\rightarrow\infty} \mu(\bigcup\limits_{k = n}^{\infty} E_k) &&\text{(substituting back)} \\ 
    &\leq \lim_{n\rightarrow\infty}(\sum\limits_{k = n}^{\infty} \mu(E_k)) &&\text{(by countable subadditivity property of measure)} \\
    &= 0.
\end{align*}
The last line above follows intuitively. We are given that $\sum\limits_{n=1}^{\infty} \mu(E_n)$ converges, so the tail end of the sum must tend to 0. 
More formally, but briefly: by definition, the corresponding partial sums $S_1, S_2, S_3, ...$ must converge, say, to $L$. Thus, for any $\epsilon > 0$, 
there exists some $N$ such that for $k - 1 = N$, we have $|S_{k-1} - L| < \epsilon$. Decomposing $L$ into the k-1th partial sum and the remaining tail, 
we subtract the two identical partial sums. We are left with the tail $\sum\limits_{k = N}^{\infty} \mu(E_k) < \epsilon$. This is the limit in the second to last line,
and it converges to 0. To complete the proof, by non-negativity of measure, we have the limit superior bounded below by 0, too. 
Thus, $\mu(\limsup\limits_{n\rightarrow\infty} E_n) = 0$. $\qedhere$ \\
\end{proof}
\end{document}