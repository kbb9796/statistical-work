\documentclass{article}

\usepackage{amsmath,amsthm,amssymb}

\begin{document}

The inspection paradox is one of my favorite results in probability theory. It says that for a renewal process with a finite average waiting time, 
an observer will spend more time waiting for the next arrival than the average waiting time itself. For example, if a bus arrives at a bus stop 
every 5 minutes on average, someone waiting to get on the bus will wait longer than 5 minutes on average. This is certainly paradoxical, but nonetheless 
feels true from experience. I prove this fact below. 

\begin{proof}
    To start, we define some variables. Let $X_t$ be the value of the renewal process at time $t$; let $J_{X_t}$ be the time of the $X_t$'th arrival, 
    and $S_{X_t}$ the holding time for the $X_t$'th arrival. Now, consider time $t$ in the interval $[J_{X_t}, J_{{X_t}+1}]$, i.e. we have observed 
    $X_t$ but not $X_t + 1$. Then, we have
  \begin{align*}
    P(S_{{X_t}+1} > x) = \int_{0}^{\inf} P(S_{{X_t}+1} > x | J_{X_t} = s) f_{J_{X_t}} ds \\
    &= \int_{0}^{\inf} P(S_{{X_t}+1} > x | S_{{X_t}+1} > t - s) f_{J_{X_t}} ds \\
    &= \int_{0}^{\inf} \frac{P(S_{{X_t}+1} > x, S_{{X_t}+1} > t - s)}{P(S_{{X_t}+1} > t - s)} f_{J_{X_t}} ds \\
    &= \int_{0}^{\inf} \frac{1 - F(max{x, t -s})}{1 - F(t -s)} f_{J_{X_t}} ds \\
    &= \int_{0}^{\inf} min{\frac{1 - F(x)}{1 - F(t - s)}, \frac{1 - F(t -s )}{1 - F(t - s)}} f_{J_{X_t}} ds \\
    &= \int_{0}^{\inf} min{\frac{1 - F(x)}{1 - F(t - s)}, 1} f_{J_{X_t}} ds \\
    &\geq \int_{0}^{\inf} (1 - F(x)) f_{J_{X_t}} ds \\
    &= 1 - F(x)
    &= P(S_{1} > x)
  \end{align*}
\end{proof}

\end{document}